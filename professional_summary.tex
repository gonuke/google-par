\section{Professional Summary}

Wilson, with Emeritus Professor Corradini and consultant Dr. Thomas Palmieri,
published a report on the "Analysis of the Case for FederalSupport of
Micro-Scale Nuclear Reactors to Provide Secure Power at U.S. Government
Installations".  This report was the culmination of a study to understand the
potential role of the U.S. Government as an early consumer of novel microscale
reactor technology. The report explored benefits to both resilience and
economics, both as part of a conventional microgrid and as part of a low-carbon
system. The U.S. Government has filled this role as a technology policy for
other technologies.

Wilson was elected as the Chair-elect of the Nuclear Engineering Department
Heads Organization, where he will have new opportunities to influence the
nuclear engineering education and federal research funding that supports such 
programs.

After delays resulting from COVID, two students completed and defended their PhD
dissertations.  Arrielle Opotowsky explored data science approaches to determine
the origin of special nuclear material in a nuclear forensics context, with a
particular interest in the success of such methods as the tools to analyze those
materials become less accurate.  She chose to write a chapter aimed at the
public, including interesting custom-drawn images relating the special nuclear
material to a steak. Philip Britt developed novel approaches to variance
reduction that relied on the continuous energy adjoing Monte Carlo capabilities
of the FRENSIE software previously developed at UW-Madison.