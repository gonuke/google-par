\section{Professional Summary}

Wilson's research in nuclear security has brought an increased focus on data
science methods, with one student exploring methods to determine the origin of
interdicted nuclear material (forensics) and another student using different
methods to analyze environmental radiation detector counts to determine
anomalies (non-proliferation).  This latter effort has spurred a collaboration
with Rob Nowak (ECE) who is an expert in the fundamental mathematics of machine
learning algorithms and their application to a variety of problems.  Research in
fusion neutronics has pivoted away from high-fidelity and computationally
intense shutdown dose rate simulation to methodologies that support more rapid
design iteration earlier in the design process. 

As chair of the Fusion Energy Division of the American Nuclear Society, Wilson
had the opportunity to participate as an ex-officio member of the Fusion Energy
Sciences Advisory Committee at an important time when adopted a strategic plan
for the Office of Fusion Energy Sciences, based on a multi-year effort to
collect and align the interests of the plasma physics and fusion communities.

As department chair, much of Wilson's time was devoted to facilitating the
Engineering Physics Department's response to the COVID-19 pandemic, beginning
with an urgent transition to online instruction in March 2020, and through a
Fall semester that began with strong in-person elements only to face a 2 week
pivot to full remote instruction.  A host of administrative tasks were necessary
to facilitate a rapid return to research through the summer of 2020, and ensure
smooth campus operations throughout the remainder of the year.