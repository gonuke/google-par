\section{Professional Summary}

With nuclear energy becoming a topic of great political and public interest in the State of
Wisconsin, Wilson has had increased opportunities to engage with various
stakeholders to learn about the advances in fission and fusion technology and
the opportunities for the State.  Since the late Spring, he has provided formal
testimony in legislative hearings, participated in Q\&A sessions with individual
lawmakers, been interviewed by journalists from around the country, and spoken
about the future of nuclear energy in public forums.  This activity culminated 
with the Public Service Commission of Wisconsin reaching out to lead the 
Wisconsin Nuclear Energy Siting Study during 2026.

At the same time, there has been substantial return of interest in the Cyclus
fuel cycle modeling tools pioneered in Wilson's group. Not only is there renewed
interest in this for technical innovations and economic challenges in the
fission fuel cycle, there is growing interest in applying this to the fusion
fuel cycle to understand the challenges of tritium management across the complete
fusion energy enterprise from three facets: safety, sustainability, and security.
