As the first chair of the College of Engineering Promotions and Tenure
Committee, I needed to establish policies and procedures for completing fair
reviews of Third Year reviews of assistant professors, tenure cases and
promotions to full professor.  All of this was completed in the spirit of
continuous improvement for departments, rather than simply providing an extra
hurdle for departments.  In addition to developing a streamlined approach for
performing these reviews, the committee compiled information about mentoring
policies in each department.

With faculty from 5 other institutions, I joined a project based at UIUC to
invoke collaborative software development practices for curriculum
development.  Dubbed the Nuclear Engineering Curriculum Exchange (NEC-X), this
effort will develop a network of individual nodes of learning content.  Each
node must contain:
\begin{itemize}
\item a learning objective,
\item a full list of nodes upon which it depends, and
\item an assessment exercise that tests the learning objective.
\end{itemize}
All nodes will be subject to peer review upon addition, and may be improved
upon, with changes further subject to peer review.  Nodes may contain many
other types of content and are anticipated to include lecture materials in
various forms as well as a number of assesment exercises beyond the minimum.
When reasonably complete, an instructor will be able to define a set of nodes
that will make up a course, see all the pre-requisite material, and decide
which ones must be included in this course and which should be covered in
pre-requisite courses.  The goal is that peer review and collaborative
improvement will result in high quality and easily reusable learning
materials.
